%%%%%%%%%%%%%%%%%%%%%%%%%%%%%%%%%%%%%%%%%%%%%%%%%%%%%%%%%%%%%%%%%%%%%%%%%%%%%%%%%%%%%%%%%%%%%%%%
%
% CSCI 1430 Written Question Template
%
% This is a LaTeX document. LaTeX is a markup language for producing documents.
% Your task is to answer the questions by filling out this document, then to
% compile this into a PDF document.
%
% TO COMPILE:
% > pdflatex thisfile.tex

% If you do not have LaTeX, your options are:
% - VSCode extension: https://marketplace.visualstudio.com/items?itemName=James-Yu.latex-workshop
% - Online Tool: https://www.overleaf.com/ - most LaTeX packages are pre-installed here (e.g., \usepackage{}).
% - Personal laptops (all common OS): http://www.latex-project.org/get/ 
%
% If you need help with LaTeX, please come to office hours.
% Or, there is plenty of help online:
% https://en.wikibooks.org/wiki/LaTeX
%
% Good luck!
% The CSCI 1430 staff
%
%%%%%%%%%%%%%%%%%%%%%%%%%%%%%%%%%%%%%%%%%%%%%%%%%%%%%%%%%%%%%%%%%%%%%%%%%%%%%%%%%%%%%%%%%%%%%%%%
%
% How to include two graphics on the same line:
% 
% \includegraphics[width=0.49\linewidth]{yourgraphic1.png}
% \includegraphics[width=0.49\linewidth]{yourgraphic2.png}
%
% How to include equations:
%
% \begin{equation}
% y = mx+c
% \end{equation}
% 
%%%%%%%%%%%%%%%%%%%%%%%%%%%%%%%%%%%%%%%%%%%%%%%%%%%%%%%%%%%%%%%%%%%%%%%%%%%%%%%%%%%%%%%%%%%%%%

\documentclass[11pt]{article}

\usepackage[english]{babel}
\usepackage[utf8]{inputenc}
\usepackage[colorlinks = true,
            linkcolor = blue,
            urlcolor  = blue]{hyperref}
\usepackage[a4paper,margin=1.5in]{geometry}
\usepackage{stackengine,graphicx}
\usepackage{fancyhdr}
\setlength{\headheight}{15pt}
\usepackage{microtype}
\usepackage{times}
\usepackage{booktabs}

% From https://ctan.org/pkg/matlab-prettifier
\usepackage[numbered,framed]{matlab-prettifier}

\frenchspacing
\setlength{\parindent}{0cm} % Default is 15pt.
\setlength{\parskip}{0.3cm plus1mm minus1mm}

\pagestyle{fancy}
\fancyhf{}
\lhead{Final Project Progress Report}
\rhead{CSCI 1430}
\rfoot{\thepage}

\date{}

\title{\vspace{-1cm}Final Project Progress Report}

\begin{document}
\maketitle
\vspace{-1cm}
\thispagestyle{fancy}
**Important**: In your report, please
1) Make it very clear who on the team contributed what, and 
2) Include the dates of at least *two* meetings you had with your mentor.
\textbf{Team name: \emph{The Playmakers}}\\
\textbf{TA name: \emph{Joel Manasseh}} \\
\textbf{Meeting Date(s): \emph{4/24/24}}


\emph{Note:} when submitting this document to Gradescope, make sure to add all other team members to the submission. This can be done on the submission page after uploading.

\section*{Progress Report Instructions}

Before writing your progress report, you should have met with your TA and talked through your progress.

\subsection*{Team contributions}

Please describe in one paragraph (3--4 sentences) per team member what each of you contributed to the project so far.
\begin{description}
\item[Chenhao Lu] Investigated different methods used for the player detection and tracking task, e.g. OpenCV, YOLO, etc. Waiting for the data collection to be done before experimenting with the approaches.
\item[Ruida Zeng] Studied methods that can be used to perform the computer vision task of training the NFL plays data, this includes things like parsing, video compression, and frame reduction. Also looked at ways we can train the data besides using Google Colab (potential limited GPU resources) such as Amazon S3/EC2 or Microsoft Azure. Analyzed and approximated training time on the amount of data points.
\item[Atif Khan]  Researched different methods used for wide receiver detection and route tracking, such as, YOLO (You Only Look Once), SSD (Single Shot Detector), Faster R-CNN (Region-based Convolutional Neural Network). Most likely will go with YOLO since it is extremely fast and accurate. It divides the image into a grid and predicts bounding boxes and probabilities for each grid cell which is especially useful for real-time object detection.
\item [John Michael Slezak] Created a pipeline for data collection and was able to make 160 labeled images for use in analysis. Now that I have the process down for collecting and labeling the images I hope for it to go faster and make it to 1000 images. I am using Roboflow for the labeling.
\end{description}

\end{document}
